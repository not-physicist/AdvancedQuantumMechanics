\documentclass{scrartcl}
\KOMAoptions{fontsize=12pt, paper=a4}     % Schriftgröße und Papierformat setzen
\KOMAoptions{DIV=13}                      % Parameter mit dem man den Seitenrand ändern kann
% \KOMAoptions{listof=totoc}				  %	Abbildungs- und 	Tabellenverzeichnisse 

% Hier werden einige Pakete eingebunden
\usepackage[utf8]{inputenc}               % Direkte Eingabe von ä usw.
\usepackage[T1]{fontenc}                  % Font Kodierung für die Ausgabe
\usepackage{babel}                        % Verschiedenste sprach-spezifische Extras
\usepackage[autostyle=true]{csquotes}     % Intelligente Anführungszeichen
\usepackage[varg]{txfonts}  			  %	Times-like fonts in support of mathematics
\usepackage{siunitx}   	  				  % Intelligentes Setzen von Zahlen und Einheiten
\usepackage{enumitem}				      %	extra enumerate options

\renewcommand{\familydefault}{\rmdefault} % font to sans serif

%import external graphics and where to find these
\usepackage{graphicx}					  
\graphicspath{{figs/}}

% Biblographie mit Biber
\RequirePackage[backend=biber, style=numeric]{biblatex}
\addbibresource{refs.bib}

%hyperref. Das Paket hypcap sorgt dafür, dass Verweise auf Abbildungen besser gesetzt sind, als wenn Sie nur hyperref verwenden. 
\usepackage{hyperref}
\RequirePackage[all]{hypcap}

%There are a number of symbols defined inside txfonts that are also defined in amsmath
% so you can just make these available again
\let\iint\relax
\let\iiint\relax
\let\iiiint\relax
\let\idotsint\relax
\usepackage{amsmath}
\usepackage{physics}
\usepackage{mathtools}
\usepackage{braket}
\usepackage{listings}  % Zum Einbinden von Programmcode verwenden wir das listings-Paket
\usepackage{xcolor}  % Für Syntax-Highlighting:
% Die folgenden listings-Einstellungen sind nötig, um
% deutsche Umlaute und die Tilde (~) in listings-Umgebungen
% verwenden zu können.
\lstset{basicstyle=\ttfamily,    
		literate={~} {$\sim$}{1} % set tilde as a literal
		{ö}{{\"o}}1
		{ä}{{\"a}}1
		{ü}{{\"u}}1
		{ß}{{\ss}}1
		{Ö}{{\"O}}1
		{Ä}{{\"A}}1
		{Ü}{{\"U}}1
}
% Some styles for code blocks
\definecolor{codebg}{HTML}{EEEEEE}
\definecolor{codeframe}{HTML}{CCCCCC}
\definecolor{gree}{HTML}{1bb703}
\lstset{% Keine besondere Markierung für Leerzeichen in Codes
		showspaces=false,               
		showstringspaces=false,         
		% Farebn für Code-Kommentare und Schlüsselworte:
		commentstyle=\color{red},       % comment style
		keywordstyle=\color{blue},      % keyword style
		backgroundcolor=\color{codebg},
		rulecolor=\color{codeframe},
		showtabs=false,
		tabsize=4,
		numbers=left,
		numbersep=15pt,
		frame=single,
		framesep=10pt,
		stringstyle=\color{gree},
		basicstyle=\footnotesize,
		breakatwhitespace=false,
}

% This will set fancy headings to the top of the page. The page number will be
% accompanied by the total number of pages. That way, you will know if any page
% is missing.
% If you do not want this for your document, you can just use
% ``\pagestyle{plain}``.
\usepackage{scrpage2}
\usepackage{lastpage}
\pagestyle{scrheadings}
\automark{section}
\cfoot{\footnotesize{Seite \thepage\ / \pageref*{LastPage}}}
\chead{}
\ihead{}
\ohead{\rightmark}
\setheadsepline{.4pt}

%symbols for footnotes
\usepackage[symbol]{footmisc}
\renewcommand{\thefootnote}{\fnsymbol{footnote}}

% {figure}[H]
\usepackage{here}

% toprule and etc.
\usepackage{booktabs}

%%%%%%%%%%%%%%%%%%%%%%%%%% NEW COMMAND SECTION %%%%%%%%%%%%%%%%%%%%
%define equal
\newcommand{\defeq}{\vcentcolon =} 
\newcommand{\eqdef}{= \vcentcolon}
\newcommand{\euler}{\mathrm{e}}

%%%%%%%%%%%%%%%%%%%%%%%%% OPTIONAL %%%%%%%%%%%%%%%%%%%%%%%%%%%%%%%%
% Using the package makecell to force a line break inside a table cell
%\usepackage{makecell}

%set the line spacing
%\renewcommand{\baselinestretch}{1} 

%section number to roman numbers
%\renewcommand\thesection{\Roman{section}}

%Exclude figures
%\usepackage{comment}
%\excludecomment{figure}
%\let\endfigure\relax

%Centering style for section titles
%\usepackage{sectsty}
%\allsectionsfont{\centering}
